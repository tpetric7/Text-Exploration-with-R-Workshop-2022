% Options for packages loaded elsewhere
\PassOptionsToPackage{unicode}{hyperref}
\PassOptionsToPackage{hyphens}{url}
%
\documentclass[
  letterpaper,
]{scrbook}

\usepackage{amsmath,amssymb}
\usepackage{lmodern}
\usepackage{iftex}
\ifPDFTeX
  \usepackage[T1]{fontenc}
  \usepackage[utf8]{inputenc}
  \usepackage{textcomp} % provide euro and other symbols
\else % if luatex or xetex
  \usepackage{unicode-math}
  \defaultfontfeatures{Scale=MatchLowercase}
  \defaultfontfeatures[\rmfamily]{Ligatures=TeX,Scale=1}
\fi
% Use upquote if available, for straight quotes in verbatim environments
\IfFileExists{upquote.sty}{\usepackage{upquote}}{}
\IfFileExists{microtype.sty}{% use microtype if available
  \usepackage[]{microtype}
  \UseMicrotypeSet[protrusion]{basicmath} % disable protrusion for tt fonts
}{}
\makeatletter
\@ifundefined{KOMAClassName}{% if non-KOMA class
  \IfFileExists{parskip.sty}{%
    \usepackage{parskip}
  }{% else
    \setlength{\parindent}{0pt}
    \setlength{\parskip}{6pt plus 2pt minus 1pt}}
}{% if KOMA class
  \KOMAoptions{parskip=half}}
\makeatother
\usepackage{xcolor}
\setlength{\emergencystretch}{3em} % prevent overfull lines
\setcounter{secnumdepth}{5}
% Make \paragraph and \subparagraph free-standing
\ifx\paragraph\undefined\else
  \let\oldparagraph\paragraph
  \renewcommand{\paragraph}[1]{\oldparagraph{#1}\mbox{}}
\fi
\ifx\subparagraph\undefined\else
  \let\oldsubparagraph\subparagraph
  \renewcommand{\subparagraph}[1]{\oldsubparagraph{#1}\mbox{}}
\fi

\usepackage{color}
\usepackage{fancyvrb}
\newcommand{\VerbBar}{|}
\newcommand{\VERB}{\Verb[commandchars=\\\{\}]}
\DefineVerbatimEnvironment{Highlighting}{Verbatim}{commandchars=\\\{\}}
% Add ',fontsize=\small' for more characters per line
\usepackage{framed}
\definecolor{shadecolor}{RGB}{241,243,245}
\newenvironment{Shaded}{\begin{snugshade}}{\end{snugshade}}
\newcommand{\AlertTok}[1]{\textcolor[rgb]{0.68,0.00,0.00}{#1}}
\newcommand{\AnnotationTok}[1]{\textcolor[rgb]{0.37,0.37,0.37}{#1}}
\newcommand{\AttributeTok}[1]{\textcolor[rgb]{0.40,0.45,0.13}{#1}}
\newcommand{\BaseNTok}[1]{\textcolor[rgb]{0.68,0.00,0.00}{#1}}
\newcommand{\BuiltInTok}[1]{\textcolor[rgb]{0.00,0.23,0.31}{#1}}
\newcommand{\CharTok}[1]{\textcolor[rgb]{0.13,0.47,0.30}{#1}}
\newcommand{\CommentTok}[1]{\textcolor[rgb]{0.37,0.37,0.37}{#1}}
\newcommand{\CommentVarTok}[1]{\textcolor[rgb]{0.37,0.37,0.37}{\textit{#1}}}
\newcommand{\ConstantTok}[1]{\textcolor[rgb]{0.56,0.35,0.01}{#1}}
\newcommand{\ControlFlowTok}[1]{\textcolor[rgb]{0.00,0.23,0.31}{#1}}
\newcommand{\DataTypeTok}[1]{\textcolor[rgb]{0.68,0.00,0.00}{#1}}
\newcommand{\DecValTok}[1]{\textcolor[rgb]{0.68,0.00,0.00}{#1}}
\newcommand{\DocumentationTok}[1]{\textcolor[rgb]{0.37,0.37,0.37}{\textit{#1}}}
\newcommand{\ErrorTok}[1]{\textcolor[rgb]{0.68,0.00,0.00}{#1}}
\newcommand{\ExtensionTok}[1]{\textcolor[rgb]{0.00,0.23,0.31}{#1}}
\newcommand{\FloatTok}[1]{\textcolor[rgb]{0.68,0.00,0.00}{#1}}
\newcommand{\FunctionTok}[1]{\textcolor[rgb]{0.28,0.35,0.67}{#1}}
\newcommand{\ImportTok}[1]{\textcolor[rgb]{0.00,0.46,0.62}{#1}}
\newcommand{\InformationTok}[1]{\textcolor[rgb]{0.37,0.37,0.37}{#1}}
\newcommand{\KeywordTok}[1]{\textcolor[rgb]{0.00,0.23,0.31}{#1}}
\newcommand{\NormalTok}[1]{\textcolor[rgb]{0.00,0.23,0.31}{#1}}
\newcommand{\OperatorTok}[1]{\textcolor[rgb]{0.37,0.37,0.37}{#1}}
\newcommand{\OtherTok}[1]{\textcolor[rgb]{0.00,0.23,0.31}{#1}}
\newcommand{\PreprocessorTok}[1]{\textcolor[rgb]{0.68,0.00,0.00}{#1}}
\newcommand{\RegionMarkerTok}[1]{\textcolor[rgb]{0.00,0.23,0.31}{#1}}
\newcommand{\SpecialCharTok}[1]{\textcolor[rgb]{0.37,0.37,0.37}{#1}}
\newcommand{\SpecialStringTok}[1]{\textcolor[rgb]{0.13,0.47,0.30}{#1}}
\newcommand{\StringTok}[1]{\textcolor[rgb]{0.13,0.47,0.30}{#1}}
\newcommand{\VariableTok}[1]{\textcolor[rgb]{0.07,0.07,0.07}{#1}}
\newcommand{\VerbatimStringTok}[1]{\textcolor[rgb]{0.13,0.47,0.30}{#1}}
\newcommand{\WarningTok}[1]{\textcolor[rgb]{0.37,0.37,0.37}{\textit{#1}}}

\providecommand{\tightlist}{%
  \setlength{\itemsep}{0pt}\setlength{\parskip}{0pt}}\usepackage{longtable,booktabs,array}
\usepackage{calc} % for calculating minipage widths
% Correct order of tables after \paragraph or \subparagraph
\usepackage{etoolbox}
\makeatletter
\patchcmd\longtable{\par}{\if@noskipsec\mbox{}\fi\par}{}{}
\makeatother
% Allow footnotes in longtable head/foot
\IfFileExists{footnotehyper.sty}{\usepackage{footnotehyper}}{\usepackage{footnote}}
\makesavenoteenv{longtable}
\usepackage{graphicx}
\makeatletter
\def\maxwidth{\ifdim\Gin@nat@width>\linewidth\linewidth\else\Gin@nat@width\fi}
\def\maxheight{\ifdim\Gin@nat@height>\textheight\textheight\else\Gin@nat@height\fi}
\makeatother
% Scale images if necessary, so that they will not overflow the page
% margins by default, and it is still possible to overwrite the defaults
% using explicit options in \includegraphics[width, height, ...]{}
\setkeys{Gin}{width=\maxwidth,height=\maxheight,keepaspectratio}
% Set default figure placement to htbp
\makeatletter
\def\fps@figure{htbp}
\makeatother
\newlength{\cslhangindent}
\setlength{\cslhangindent}{1.5em}
\newlength{\csllabelwidth}
\setlength{\csllabelwidth}{3em}
\newlength{\cslentryspacingunit} % times entry-spacing
\setlength{\cslentryspacingunit}{\parskip}
\newenvironment{CSLReferences}[2] % #1 hanging-ident, #2 entry spacing
 {% don't indent paragraphs
  \setlength{\parindent}{0pt}
  % turn on hanging indent if param 1 is 1
  \ifodd #1
  \let\oldpar\par
  \def\par{\hangindent=\cslhangindent\oldpar}
  \fi
  % set entry spacing
  \setlength{\parskip}{#2\cslentryspacingunit}
 }%
 {}
\usepackage{calc}
\newcommand{\CSLBlock}[1]{#1\hfill\break}
\newcommand{\CSLLeftMargin}[1]{\parbox[t]{\csllabelwidth}{#1}}
\newcommand{\CSLRightInline}[1]{\parbox[t]{\linewidth - \csllabelwidth}{#1}\break}
\newcommand{\CSLIndent}[1]{\hspace{\cslhangindent}#1}

\usepackage{makeidx}
\makeindex
\makeatletter
\makeatother
\makeatletter
\@ifpackageloaded{bookmark}{}{\usepackage{bookmark}}
\makeatother
\makeatletter
\@ifpackageloaded{caption}{}{\usepackage{caption}}
\AtBeginDocument{%
\ifdefined\contentsname
  \renewcommand*\contentsname{Table of contents}
\else
  \newcommand\contentsname{Table of contents}
\fi
\ifdefined\listfigurename
  \renewcommand*\listfigurename{List of Figures}
\else
  \newcommand\listfigurename{List of Figures}
\fi
\ifdefined\listtablename
  \renewcommand*\listtablename{List of Tables}
\else
  \newcommand\listtablename{List of Tables}
\fi
\ifdefined\figurename
  \renewcommand*\figurename{Figure}
\else
  \newcommand\figurename{Figure}
\fi
\ifdefined\tablename
  \renewcommand*\tablename{Table}
\else
  \newcommand\tablename{Table}
\fi
}
\@ifpackageloaded{float}{}{\usepackage{float}}
\floatstyle{ruled}
\@ifundefined{c@chapter}{\newfloat{codelisting}{h}{lop}}{\newfloat{codelisting}{h}{lop}[chapter]}
\floatname{codelisting}{Listing}
\newcommand*\listoflistings{\listof{codelisting}{List of Listings}}
\makeatother
\makeatletter
\@ifpackageloaded{caption}{}{\usepackage{caption}}
\@ifpackageloaded{subcaption}{}{\usepackage{subcaption}}
\makeatother
\makeatletter
\@ifpackageloaded{tcolorbox}{}{\usepackage[many]{tcolorbox}}
\makeatother
\makeatletter
\@ifundefined{shadecolor}{\definecolor{shadecolor}{rgb}{.97, .97, .97}}
\makeatother
\makeatletter
\makeatother
\ifLuaTeX
  \usepackage{selnolig}  % disable illegal ligatures
\fi
\IfFileExists{bookmark.sty}{\usepackage{bookmark}}{\usepackage{hyperref}}
\IfFileExists{xurl.sty}{\usepackage{xurl}}{} % add URL line breaks if available
\urlstyle{same} % disable monospaced font for URLs
\hypersetup{
  pdftitle={Delavnica za delo z lastnimi besedilnimi zbirkami v računalniškem jeziku R},
  pdfauthor={Teodor Petrič},
  hidelinks,
  pdfcreator={LaTeX via pandoc}}

\title{Delavnica za delo z lastnimi besedilnimi zbirkami v računalniškem
jeziku R}
\usepackage{etoolbox}
\makeatletter
\providecommand{\subtitle}[1]{% add subtitle to \maketitle
  \apptocmd{\@title}{\par {\large #1 \par}}{}{}
}
\makeatother
\subtitle{Center za jezikoslovne raziskave Filozofske fakultete Univerze
v Mariboru}
\author{Teodor Petrič}
\date{12/12/22}

\begin{document}
\frontmatter
\maketitle
\ifdefined\Shaded\renewenvironment{Shaded}{\begin{tcolorbox}[interior hidden, sharp corners, enhanced, breakable, borderline west={3pt}{0pt}{shadecolor}, frame hidden, boxrule=0pt]}{\end{tcolorbox}}\fi

\renewcommand*\contentsname{Table of contents}
{
\setcounter{tocdepth}{2}
\tableofcontents
}
\mainmatter
\bookmarksetup{startatroot}

\hypertarget{section}{%
\chapter*{.}\label{section}}
\addcontentsline{toc}{chapter}{.}

\markboth{.}{.}

\begin{Shaded}
\begin{Highlighting}[]
\NormalTok{knitr}\SpecialCharTok{::}\FunctionTok{include\_graphics}\NormalTok{(}\StringTok{"pictures/Diapozitiv1.PNG"}\NormalTok{)}
\end{Highlighting}
\end{Shaded}

\begin{figure}[H]

{\centering \includegraphics[width=1\textwidth,height=\textheight]{./pictures/Diapozitiv1.PNG}

}

\end{figure}

\bookmarksetup{startatroot}

\hypertarget{sec-vorwort}{%
\chapter*{Vorwort}\label{sec-vorwort}}
\addcontentsline{toc}{chapter}{Vorwort}

\markboth{Vorwort}{Vorwort}

Dieses Buch ist eine Einführung in die Phonologie der deutschen Sprache,
und zwar unter besonderer Berücksichtigung des Slowenischen im Vergleich
zum Deutschen.

\texttt{Quarto\ Book} \url{https://quarto.org/}

\part{Prvi koraki}

\hypertarget{sec-delavnica1}{%
\chapter{Slovenska literatura 1968}\label{sec-delavnica1}}

\hypertarget{sec-delavnica2}{%
\chapter{Quanteda korpus}\label{sec-delavnica2}}

\begin{Shaded}
\begin{Highlighting}[]
\NormalTok{packages }\OtherTok{=} \FunctionTok{c}\NormalTok{(}\StringTok{"tidyverse"}\NormalTok{, }\StringTok{"tidytext"}\NormalTok{, }\StringTok{"janitor"}\NormalTok{, }\StringTok{"scales"}\NormalTok{, }\StringTok{"widyr"}\NormalTok{, }
             \StringTok{"quanteda"}\NormalTok{, }\StringTok{"quanteda.textplots"}\NormalTok{, }\StringTok{"quanteda.textstats"}\NormalTok{,}
             \StringTok{"wordcloud2"}\NormalTok{, }\StringTok{"ggwordcloud"}\NormalTok{, }\StringTok{"udpipe"}\NormalTok{, }\StringTok{"syuzhet"}\NormalTok{,}
             \StringTok{"ggtext"}\NormalTok{, }\StringTok{"corpustools"}\NormalTok{, }\StringTok{"xml2"}\NormalTok{, }\StringTok{"XML"}\NormalTok{, }\StringTok{"rvest"}\NormalTok{, }
             \StringTok{"readtext"}\NormalTok{, }\StringTok{"readxl"}\NormalTok{, }\StringTok{"writexl"}\NormalTok{, }\StringTok{"xlsx"}\NormalTok{)}

\CommentTok{\# Install packages not yet installed}
\NormalTok{installed\_packages }\OtherTok{\textless{}{-}}\NormalTok{ packages }\SpecialCharTok{\%in\%} \FunctionTok{rownames}\NormalTok{(}\FunctionTok{installed.packages}\NormalTok{())}
\ControlFlowTok{if}\NormalTok{ (}\FunctionTok{any}\NormalTok{(installed\_packages }\SpecialCharTok{==} \ConstantTok{FALSE}\NormalTok{)) \{}
  \FunctionTok{install.packages}\NormalTok{(packages[}\SpecialCharTok{!}\NormalTok{installed\_packages])}
\NormalTok{\}}

\CommentTok{\# Packages loading}
\FunctionTok{invisible}\NormalTok{(}\FunctionTok{lapply}\NormalTok{(packages, library, }\AttributeTok{character.only =} \ConstantTok{TRUE}\NormalTok{))}
\end{Highlighting}
\end{Shaded}

Knjižnica \texttt{quanteda} ima nekaj zelo priročnih funkcij za prikaz
konkordance in različne grafike.

S knjižnico \texttt{quanteda} bomo ustvarili korpus (\texttt{corpus}),
ga razdrobili na manjše jezikovne enote (\texttt{tokens}) in pripravili
matriko (\texttt{dfm}), ki vključuje besedila v vrsticah in besede (oz.
tokens) v stolpcih.

To so gradniki, ki jih potrebujemo za statistične in grafične funkcije v
knjižnici \texttt{quanteda}.

\begin{Shaded}
\begin{Highlighting}[]
\NormalTok{df }\OtherTok{\textless{}{-}} \FunctionTok{read\_csv2}\NormalTok{(}\StringTok{"data/slovenska\_literatura\_1968.csv"}\NormalTok{)}
\NormalTok{corp }\OtherTok{\textless{}{-}} \FunctionTok{corpus}\NormalTok{(df, }\AttributeTok{text\_field =} \StringTok{"text"}\NormalTok{, }\AttributeTok{docid\_field =} \StringTok{"doc\_id"}\NormalTok{)}
\NormalTok{toks }\OtherTok{\textless{}{-}} \FunctionTok{tokens}\NormalTok{(corp, }\AttributeTok{remove\_numbers =} \ConstantTok{TRUE}\NormalTok{, }\AttributeTok{remove\_punct =}\NormalTok{ T,}
               \AttributeTok{remove\_symbols =}\NormalTok{ T, }\AttributeTok{remove\_url =}\NormalTok{ T, }
               \AttributeTok{remove\_separators =} \ConstantTok{FALSE}\NormalTok{)}
\NormalTok{mat }\OtherTok{\textless{}{-}} \FunctionTok{dfm}\NormalTok{(toks, }\AttributeTok{padding =} \ConstantTok{TRUE}\NormalTok{)}
\end{Highlighting}
\end{Shaded}

\part{Že tečemo}

\hypertarget{sec-delavnica3}{%
\chapter{Delavnica 3}\label{sec-delavnica3}}

\hypertarget{sec-delavnica4}{%
\chapter{Delavnica 4}\label{sec-delavnica4}}

\hypertarget{xml}{%
\section{XML}\label{xml}}

\begin{Shaded}
\begin{Highlighting}[]
\NormalTok{packages }\OtherTok{=} \FunctionTok{c}\NormalTok{(}\StringTok{"tidyverse"}\NormalTok{, }\StringTok{"tidytext"}\NormalTok{, }\StringTok{"janitor"}\NormalTok{, }\StringTok{"scales"}\NormalTok{, }\StringTok{"widyr"}\NormalTok{, }
             \StringTok{"quanteda"}\NormalTok{, }\StringTok{"quanteda.textplots"}\NormalTok{, }\StringTok{"quanteda.textstats"}\NormalTok{,}
             \StringTok{"wordcloud2"}\NormalTok{, }\StringTok{"ggwordcloud"}\NormalTok{, }\StringTok{"udpipe"}\NormalTok{, }\StringTok{"syuzhet"}\NormalTok{,}
             \StringTok{"ggtext"}\NormalTok{, }\StringTok{"corpustools"}\NormalTok{, }\StringTok{"xml2"}\NormalTok{, }\StringTok{"XML"}\NormalTok{, }\StringTok{"rvest"}\NormalTok{, }
             \StringTok{"readtext"}\NormalTok{, }\StringTok{"readxl"}\NormalTok{, }\StringTok{"writexl"}\NormalTok{, }\StringTok{"xlsx"}\NormalTok{)}

\CommentTok{\# Install packages not yet installed}
\NormalTok{installed\_packages }\OtherTok{\textless{}{-}}\NormalTok{ packages }\SpecialCharTok{\%in\%} \FunctionTok{rownames}\NormalTok{(}\FunctionTok{installed.packages}\NormalTok{())}
\ControlFlowTok{if}\NormalTok{ (}\FunctionTok{any}\NormalTok{(installed\_packages }\SpecialCharTok{==} \ConstantTok{FALSE}\NormalTok{)) \{}
  \FunctionTok{install.packages}\NormalTok{(packages[}\SpecialCharTok{!}\NormalTok{installed\_packages])}
\NormalTok{\}}

\CommentTok{\# Packages loading}
\FunctionTok{invisible}\NormalTok{(}\FunctionTok{lapply}\NormalTok{(packages, library, }\AttributeTok{character.only =} \ConstantTok{TRUE}\NormalTok{))}
\end{Highlighting}
\end{Shaded}

Datoteke v obliki xml lahko spremenimo v podatkovni niz.

S spletnega portala
\href{https://www.clarin.si/repository/xmlui/handle/11356/1491}{clarin}
smo naložili v shranili literarni korpus v obliki xml, ki poleg besedil
vsebuje tudi različne metapodatke.

Najprej ustvarimo seznam relevantnih xml datotek. Datoteke imajo pripono
\emph{vert}.

\begin{Shaded}
\begin{Highlighting}[]
\NormalTok{seznam }\OtherTok{\textless{}{-}} \FunctionTok{list.files}\NormalTok{(}\StringTok{"slovenski\_korpusi/maj68.vert/"}\NormalTok{,}
                     \AttributeTok{pattern =} \StringTok{"*.vert"}\NormalTok{,}
                     \AttributeTok{full.names =} \ConstantTok{TRUE}\NormalTok{)}
\NormalTok{wd }\OtherTok{\textless{}{-}} \FunctionTok{getwd}\NormalTok{()}
\NormalTok{pot }\OtherTok{\textless{}{-}} \FunctionTok{paste0}\NormalTok{(wd, }\StringTok{"/"}\NormalTok{, seznam)}

\FunctionTok{head}\NormalTok{(pot, }\DecValTok{3}\NormalTok{)}
\end{Highlighting}
\end{Shaded}

\begin{verbatim}
[1] "D:/Users/teodo/Documents/R/Text-Exploration-with-R-Workshop-2022/slovenski_korpusi/maj68.vert/maj68-0001.vert"
[2] "D:/Users/teodo/Documents/R/Text-Exploration-with-R-Workshop-2022/slovenski_korpusi/maj68.vert/maj68-0004.vert"
[3] "D:/Users/teodo/Documents/R/Text-Exploration-with-R-Workshop-2022/slovenski_korpusi/maj68.vert/maj68-0005.vert"
\end{verbatim}

https://megapteraphile.wordpress.com/2020/03/29/converting-xml-to-tibble-in-r/

\begin{Shaded}
\begin{Highlighting}[]
\FunctionTok{library}\NormalTok{(XML)}

\NormalTok{xmldf }\OtherTok{\textless{}{-}} \ConstantTok{NULL}

\ControlFlowTok{for}\NormalTok{(i }\ControlFlowTok{in} \DecValTok{1}\SpecialCharTok{:}\FunctionTok{length}\NormalTok{(pot[}\DecValTok{1}\NormalTok{]))\{}
\NormalTok{  xml\_document }\OtherTok{\textless{}{-}} \FunctionTok{read\_xml}\NormalTok{(pot[i])}
\NormalTok{  xml\_list }\OtherTok{\textless{}{-}} \FunctionTok{as\_list}\NormalTok{(xml\_document)}
\NormalTok{  xml\_tbl }\OtherTok{\textless{}{-}} \FunctionTok{as\_tibble}\NormalTok{(xml\_list)}
\NormalTok{  xml\_df }\OtherTok{=}\NormalTok{ xml\_tbl }\SpecialCharTok{\%\textgreater{}\%} 
    \FunctionTok{unnest\_longer}\NormalTok{(text)}
\NormalTok{  litxml }\OtherTok{=} \FunctionTok{bind\_rows}\NormalTok{(xmldf, xml\_df)}
\NormalTok{\}}

\NormalTok{litxml }\SpecialCharTok{\%\textgreater{}\%} \FunctionTok{head}\NormalTok{()}
\end{Highlighting}
\end{Shaded}

\begin{verbatim}
# A tibble: 6 x 2
  text             text_id
  <named list>     <chr>  
1 <named list [3]> s      
2 <named list [5]> s      
3 <named list [7]> s      
4 <named list [7]> s      
5 <named list [5]> s      
6 <named list [5]> s      
\end{verbatim}

\begin{Shaded}
\begin{Highlighting}[]
\CommentTok{\# str(litxml$text, 1)}
\end{Highlighting}
\end{Shaded}

\begin{Shaded}
\begin{Highlighting}[]
\NormalTok{lp\_wider }\OtherTok{=}\NormalTok{ litxml }\SpecialCharTok{\%\textgreater{}\%}
  \FunctionTok{filter}\NormalTok{(text\_id }\SpecialCharTok{==} \StringTok{"s"}\NormalTok{) }\SpecialCharTok{\%\textgreater{}\%}
  \FunctionTok{unnest\_wider}\NormalTok{(text) }
\end{Highlighting}
\end{Shaded}

\begin{Shaded}
\begin{Highlighting}[]
\NormalTok{lp\_df }\OtherTok{=}\NormalTok{ lp\_wider }\SpecialCharTok{\%\textgreater{}\%}
  \CommentTok{\# 1st time unnest to release the 2{-}dimension list?}
  \FunctionTok{unnest}\NormalTok{(}\AttributeTok{cols =} \FunctionTok{names}\NormalTok{(.)) }\SpecialCharTok{\%\textgreater{}\%}
  \CommentTok{\# 2nd time to nest the single list in each cell?}
  \FunctionTok{unnest}\NormalTok{(}\AttributeTok{cols =} \FunctionTok{names}\NormalTok{(.)) }\SpecialCharTok{\%\textgreater{}\%}
  \CommentTok{\# convert data type}
\NormalTok{  readr}\SpecialCharTok{::}\FunctionTok{type\_convert}\NormalTok{()}
\end{Highlighting}
\end{Shaded}

\hypertarget{try-python-pandas}{%
\section{Try Python Pandas?}\label{try-python-pandas}}

https://stackoverflow.com/questions/28259301/how-to-convert-an-xml-file-to-nice-pandas-dataframe

\begin{Shaded}
\begin{Highlighting}[]
\FunctionTok{library}\NormalTok{(rvest)}
\NormalTok{level1 }\OtherTok{\textless{}{-}} \FunctionTok{html\_element}\NormalTok{(xml\_document, }\AttributeTok{xpath =} \StringTok{"//text[@id]"}\NormalTok{)}
\NormalTok{level2 }\OtherTok{\textless{}{-}} \FunctionTok{html\_element}\NormalTok{(xml\_document, }\AttributeTok{xpath =} \StringTok{"//p[@id]"}\NormalTok{)}
\NormalTok{level3 }\OtherTok{\textless{}{-}} \FunctionTok{html\_element}\NormalTok{(xml\_document, }\AttributeTok{xpath =} \StringTok{"//s[@id]"}\NormalTok{)}
\end{Highlighting}
\end{Shaded}

\begin{Shaded}
\begin{Highlighting}[]
\NormalTok{top }\OtherTok{\textless{}{-}} \FunctionTok{as.character}\NormalTok{(level1) }\SpecialCharTok{\%\textgreater{}\%} \FunctionTok{str\_extract}\NormalTok{(}\StringTok{"}\SpecialCharTok{\textbackslash{}\textbackslash{}}\StringTok{\textless{}text .+}\SpecialCharTok{\textbackslash{}\textbackslash{}}\StringTok{\textgreater{}"}\NormalTok{)}

\NormalTok{header }\OtherTok{\textless{}{-}}\NormalTok{ top }\SpecialCharTok{\%\textgreater{}\%} 
  \FunctionTok{as\_tibble}\NormalTok{() }\SpecialCharTok{\%\textgreater{}\%} 
  \FunctionTok{separate\_rows}\NormalTok{(value, }\AttributeTok{sep =} \StringTok{\textquotesingle{}" \textquotesingle{}}\NormalTok{) }\SpecialCharTok{\%\textgreater{}\%} 
  \FunctionTok{separate}\NormalTok{(value, }\AttributeTok{into =} \FunctionTok{c}\NormalTok{(}\StringTok{"name"}\NormalTok{, }\StringTok{"value"}\NormalTok{), }\AttributeTok{sep =} \StringTok{"="}\NormalTok{) }\SpecialCharTok{\%\textgreater{}\%} 
  \FunctionTok{mutate}\NormalTok{(}\AttributeTok{name =} \FunctionTok{str\_remove}\NormalTok{(name, }\StringTok{"[\textless{}]"}\NormalTok{),}
         \AttributeTok{value =} \FunctionTok{str\_remove\_all}\NormalTok{(value, }\StringTok{\textquotesingle{}"\textquotesingle{}}\NormalTok{)) }\SpecialCharTok{\%\textgreater{}\%} 
  \FunctionTok{pivot\_wider}\NormalTok{(}\AttributeTok{names\_from =}\NormalTok{ name, }\AttributeTok{values\_from =}\NormalTok{ value) }\SpecialCharTok{\%\textgreater{}\%} 
\NormalTok{  janitor}\SpecialCharTok{::}\FunctionTok{clean\_names}\NormalTok{()}

\FunctionTok{length}\NormalTok{(header)}
\end{Highlighting}
\end{Shaded}

\begin{verbatim}
[1] 17
\end{verbatim}

\begin{Shaded}
\begin{Highlighting}[]
\NormalTok{line\_id }\OtherTok{\textless{}{-}} \FunctionTok{as.character}\NormalTok{(level2) }\SpecialCharTok{\%\textgreater{}\%} \FunctionTok{str\_extract\_all}\NormalTok{(}\StringTok{"}\SpecialCharTok{\textbackslash{}\textbackslash{}}\StringTok{\textless{}s.*}\SpecialCharTok{\textbackslash{}\textbackslash{}}\StringTok{\textgreater{}"}\NormalTok{) }\SpecialCharTok{\%\textgreater{}\%} \FunctionTok{unlist}\NormalTok{()}
\end{Highlighting}
\end{Shaded}

regex: https://www.regular-expressions.info/lookaround.html

\begin{Shaded}
\begin{Highlighting}[]
\NormalTok{cols }\OtherTok{\textless{}{-}} \ConstantTok{NULL}
\ControlFlowTok{for}\NormalTok{(i }\ControlFlowTok{in} \DecValTok{1}\SpecialCharTok{:}\DecValTok{17}\NormalTok{)\{}
\NormalTok{  col }\OtherTok{\textless{}{-}} \FunctionTok{paste}\NormalTok{(}\StringTok{"col"}\NormalTok{, i)}
\NormalTok{  cols }\OtherTok{\textless{}{-}} \FunctionTok{rbind}\NormalTok{(cols, col)}
\NormalTok{\}}

\NormalTok{data }\OtherTok{\textless{}{-}} \FunctionTok{as.character}\NormalTok{(level2) }\SpecialCharTok{\%\textgreater{}\%} 
  \FunctionTok{str\_extract\_all}\NormalTok{(}\StringTok{"}\SpecialCharTok{\textbackslash{}\textbackslash{}}\StringTok{\textless{}s.+}\SpecialCharTok{\textbackslash{}n}\StringTok{.+"}\NormalTok{) }\SpecialCharTok{\%\textgreater{}\%} 
  \FunctionTok{unlist}\NormalTok{() }\SpecialCharTok{\%\textgreater{}\%} 
  \FunctionTok{as\_tibble}\NormalTok{() }\SpecialCharTok{\%\textgreater{}\%} 
  \FunctionTok{separate}\NormalTok{(value, }\AttributeTok{into =} \FunctionTok{c}\NormalTok{(}\StringTok{"name"}\NormalTok{, }\StringTok{"value"}\NormalTok{), }
           \AttributeTok{sep =} \StringTok{"}\SpecialCharTok{\textbackslash{}\textbackslash{}}\StringTok{\textgreater{}}\SpecialCharTok{\textbackslash{}n}\StringTok{"}\NormalTok{, }\AttributeTok{extra =} \StringTok{"merge"}\NormalTok{, }\AttributeTok{fill =} \StringTok{"right"}\NormalTok{) }\SpecialCharTok{\%\textgreater{}\%} 
  \FunctionTok{mutate}\NormalTok{(}\AttributeTok{name =} \FunctionTok{str\_remove}\NormalTok{(name, }\StringTok{"}\SpecialCharTok{\textbackslash{}\textbackslash{}}\StringTok{\textless{}s "}\NormalTok{)) }\SpecialCharTok{\%\textgreater{}\%} 
  \FunctionTok{separate}\NormalTok{(value, }\AttributeTok{into =}\NormalTok{ cols, }
           \AttributeTok{sep =} \StringTok{"}\SpecialCharTok{\textbackslash{}t}\StringTok{"}\NormalTok{, }\AttributeTok{extra =} \StringTok{"merge"}\NormalTok{, }\AttributeTok{fill =} \StringTok{"right"}\NormalTok{) }\SpecialCharTok{\%\textgreater{}\%} 
  \FunctionTok{rename}\NormalTok{(}\AttributeTok{sentence\_id =}\NormalTok{ name) }\SpecialCharTok{\%\textgreater{}\%} 
  \FunctionTok{mutate}\NormalTok{(}\AttributeTok{sentence\_id =} \FunctionTok{str\_remove}\NormalTok{(sentence\_id, }\StringTok{\textquotesingle{}id="\textquotesingle{}}\NormalTok{),}
         \AttributeTok{sentence\_id =} \FunctionTok{str\_remove}\NormalTok{(sentence\_id, }\StringTok{\textquotesingle{}"\textquotesingle{}}\NormalTok{)) }\SpecialCharTok{\%\textgreater{}\%} 
  \CommentTok{\# positive lookahead: match q followed by u}
  \FunctionTok{mutate}\NormalTok{(}\AttributeTok{text\_id =} \FunctionTok{str\_extract}\NormalTok{(sentence\_id, }\StringTok{".+(?=1}\SpecialCharTok{\textbackslash{}\textbackslash{}}\StringTok{.)"}\NormalTok{),}
         \AttributeTok{text\_id =} \FunctionTok{str\_remove}\NormalTok{(text\_id, }\StringTok{"}\SpecialCharTok{\textbackslash{}\textbackslash{}}\StringTok{.$"}\NormalTok{)) }\SpecialCharTok{\%\textgreater{}\%} 
\NormalTok{  janitor}\SpecialCharTok{::}\FunctionTok{clean\_names}\NormalTok{()}
  \CommentTok{\# separate(value, into = c(names(header[1:17])), sep = "\textbackslash{}t", }
  \CommentTok{\#          extra = "merge", fill = "right")}

\FunctionTok{head}\NormalTok{(data)}
\end{Highlighting}
\end{Shaded}

\begin{verbatim}
# A tibble: 6 x 19
  sentence~1 col_1 col_2 col_3 col_4 col_5 col_6 col_7 col_8 col_9 col_10 col_11
  <chr>      <chr> <chr> <chr> <chr> <chr> <chr> <chr> <chr> <chr> <chr>  <chr> 
1 maj68-000~ Sila  sila  Ncfsn NOUN  Case~ tok1  nsubj bist~ Agpf~ ADJ    Case=~
2 maj68-000~ Zdaj  zdaj  Rgp   ADV   Degr~ tok1  advm~ oboj  Pg-n~ DET    Case=~
3 maj68-000~ Anči~ Anči~ Npfsn PROPN Case~ tok1  nsubj odlo~ Vmep~ VERB   Aspec~
4 maj68-000~ Že    že    Q     PART  _     tok1  advm~ zače~ Vmep~ VERB   Aspec~
5 maj68-000~ Tudi  tudi  Q     PART  _     tok1  advm~ spad~ Vmpr~ VERB   Aspec~
6 maj68-000~ Mami~ mami~ Ncfsn NOUN  Case~ tok1  nsubj mora~ Vmpp~ VERB   Aspec~
# ... with 7 more variables: col_12 <chr>, col_13 <chr>, col_14 <chr>,
#   col_15 <chr>, col_16 <chr>, col_17 <chr>, text_id <chr>, and abbreviated
#   variable name 1: sentence_id
\end{verbatim}

\begin{Shaded}
\begin{Highlighting}[]
\NormalTok{literatura68 }\OtherTok{\textless{}{-}}\NormalTok{ header }\SpecialCharTok{\%\textgreater{}\%} 
  \FunctionTok{left\_join}\NormalTok{(data, }\AttributeTok{by =} \StringTok{"text\_id"}\NormalTok{)}
\FunctionTok{head}\NormalTok{(literatura68)}
\end{Highlighting}
\end{Shaded}

\begin{verbatim}
# A tibble: 6 x 35
  text_id   title author birth gender year  monog~1 volnum pages text_~2 moder~3
  <chr>     <chr> <chr>  <chr> <chr>  <chr> <chr>   <chr>  <chr> <chr>   <chr>  
1 maj68-00~ Pr&#~ Mi&#x~ 1908  mo&#x~ 1964  Proble~ II/13  1&#x~ proza   odsotna
2 maj68-00~ Pr&#~ Mi&#x~ 1908  mo&#x~ 1964  Proble~ II/13  1&#x~ proza   odsotna
3 maj68-00~ Pr&#~ Mi&#x~ 1908  mo&#x~ 1964  Proble~ II/13  1&#x~ proza   odsotna
4 maj68-00~ Pr&#~ Mi&#x~ 1908  mo&#x~ 1964  Proble~ II/13  1&#x~ proza   odsotna
5 maj68-00~ Pr&#~ Mi&#x~ 1908  mo&#x~ 1964  Proble~ II/13  1&#x~ proza   odsotna
6 maj68-00~ Pr&#~ Mi&#x~ 1908  mo&#x~ 1964  Proble~ II/13  1&#x~ proza   odsotna
# ... with 24 more variables: visual <chr>, lang_type <chr>, nstd_level <chr>,
#   foreign_lang <chr>, foreign_level <chr>, facs <chr>, sentence_id <chr>,
#   col_1 <chr>, col_2 <chr>, col_3 <chr>, col_4 <chr>, col_5 <chr>,
#   col_6 <chr>, col_7 <chr>, col_8 <chr>, col_9 <chr>, col_10 <chr>,
#   col_11 <chr>, col_12 <chr>, col_13 <chr>, col_14 <chr>, col_15 <chr>,
#   col_16 <chr>, col_17 <chr>, and abbreviated variable names 1: monograph,
#   2: text_type, 3: modernism
\end{verbatim}

https://wiki.facepunch.com/gmod/Lua\_Error\_Explanation

\bookmarksetup{startatroot}

\hypertarget{summary}{%
\chapter{Summary}\label{summary}}

In summary, this book has no content whatsoever.

\begin{Shaded}
\begin{Highlighting}[]
\DecValTok{1} \SpecialCharTok{+} \DecValTok{1}
\end{Highlighting}
\end{Shaded}

\begin{verbatim}
[1] 2
\end{verbatim}

\bookmarksetup{startatroot}

\hypertarget{references}{%
\chapter*{References}\label{references}}
\addcontentsline{toc}{chapter}{References}

\markboth{References}{References}

\hypertarget{refs}{}
\begin{CSLReferences}{0}{0}
\end{CSLReferences}


\backmatter

\printindex

\end{document}
